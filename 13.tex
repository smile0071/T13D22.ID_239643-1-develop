% filepath: d:\21_school\T13D22.ID_239643-1-develop\13.tex
\documentclass[a4paper,11pt]{article}
\usepackage[T1]{fontenc}
\usepackage[utf8]{inputenc}
\usepackage{lmodern}
\title{Отчет по заданию: Quest 1 (cipher)}
\author{}
\date{}

\begin{document}
\maketitle

\section*{Цель}
Реализовать программу src/cipher.c с консольным меню и пунктом 1: чтение пути к текстовому файлу через stdin, открытие файла и вывод его содержимого. При ошибке или пустом файле выводить ``n/a''. Обеспечить сборку через Makefile и размещение исполняемого файла в папке \texttt{build}.

\section*{Реализация}
\begin{itemize}
  \item Добавлен файл \texttt{src/cipher.c}. Программа запускает цикл чтения числовых команд из stdin.
  \item При получении команды \texttt{1} программа читает следующий токен как путь к файлу.
  \item Если путь равен \texttt{"-1"}, программа завершает работу.
  \item Файл открывается в бинарном режиме, проверяется его размер (через \texttt{fseek}/\texttt{ftell}). Если размер меньше или равен нулю — выводится \texttt{n/a}.
  \item В случае успешного чтения содержимое выводится в stdout, затем выполняется дополнительный перевод строки (требование — после выполнения пункта меню должен быть перенос строки).
  \item Любая ошибка (отсутствие файла, ошибка чтения, нехватка памяти и т.п.) приводит к выводу \texttt{n/a}.
  \item Цикл завершается при вводе команды \texttt{-1} либо при некорректном вводе.
\end{itemize}

\section*{Makefile}
Добавлен Makefile с целью \texttt{cipher}, которая:
\begin{enumerate}
  \item Создаёт директорию \texttt{build} (если отсутствует).
  \item Компилирует \texttt{src/cipher.c} и кладёт бинарник в \texttt{build/cipher}.
\end{enumerate}

\section*{Особенности и замечания}
\begin{itemize}
  \item Программа минималистична: не выводит лишних сообщений, только требуемый результат или \texttt{n/a}.
  \item При чтении пути используется чтение токена (\texttt{\%s}) — ожидается, что путь не содержит пробелов (соответствует тестам).
  \item Для проверки пустоты файла используется размер файла. Пустой файл считается ошибкой и приводит к выводу \texttt{n/a}.
  \item В коде уделено внимание обработке ошибок и освобождению памяти.
  \item Для дальнейших пунктов меню (2..5) внутренняя структура уже позволяет добавлять обработчики.
\end{itemize}

\section*{Как собрать и запустить}
В корне репозитория выполнить:
\begin{verbatim}
make cipher
./build/cipher
\end{verbatim}

\section*{Тесты}
Проверено вручную на входах, соответствующих условию: подача команды 1 и имён файлов, проверка поведения при существующих/пустых/отсутствующих файлах и реакции на -1.

% filepath: d:\21_school\T13D22.ID_239643-1-develop\13.tex
\documentclass[a4paper,11pt]{article}
\usepackage[T1]{fontenc}
\usepackage[utf8]{inputenc}
\usepackage{lmodern}
\usepackage{amsmath}
\usepackage{enumitem}

\title{Отчёт по заданию: Quest 2 (Дополнение к cipher)}
\author{}
\date{\today}

\begin{document}
\maketitle

\section{Цель задачи}
Дополнить программу \texttt{src/cipher.c} реализацией пункта меню 2: приём произвольной строки текста из консоли и запись её в конец ранее загруженного файла (пункт 1). После записи требуется вывести содержимое файла в консоль. В случае ошибки или отсутствия загруженного файла необходимо вывести \texttt{n/a}. Ввод строки должен поддерживать пробелы. При вводе \texttt{-1} программа должна завершиться. Проект собирается через \texttt{Makefile}; исполняемый файл — \texttt{build/cipher}.

\section{Краткое описание выполненных шагов}
\begin{enumerate}
  \item Проанализировал поведение существующей реализации пункта 1 (чтение и вывод файла).
  \item Добавил в программу хранение текущего загруженного пути (\texttt{current\_path}) при успешном выполнении пункта 1.
  \item Реализовал обработку пункта 2:
    \begin{itemize}
      \item чтение строки через \texttt{fgets} (поддержка пробелов);
      \item безопасное удаление завершающих символов перевода строки;
      \item обработка специальной команды \texttt{"-1"} для выхода;
      \item открытие файла в режиме добавления (\texttt{"a"}), запись строки с символом новой строки;
      \item повторное чтение всего файла и вывод содержимого в stdout.
    \end{itemize}
  \item Обработал все возможные ошибки ввода/вывода: отсутствие файла, ошибки при открытии/чтении/памяти — во всех случаях выводится \texttt{n/a}.
  \item Обеспечил требование: после выполнения каждого пункта меню (кроме выхода) выполняется перенос строки.
  \item Обновил отчёт (этот документ) и добавил файл с тестовыми сценариями для ручной проверки.
\end{enumerate}

\section{Применённые методы и подходы}
\begin{itemize}
  \item Использован процедурный стиль на C (стандарт C11).
  \item Для чтения строк с пробелами использована функция \texttt{fgets}, для однострочного ввода — безопасный буфер фиксированного размера.
  \item Для определения размера файла применяются \texttt{fseek}/\texttt{ftell}; содержимое читается в выделённый динамически буфер с последующим освобождением памяти.
  \item Все операции ввода/вывода проверяются на ошибки; память освобождается при любых нештатных ситуациях.
  \item Минималистичный вывод: только содержимое файла или строка ``n/a'', без дополнительных подсказок.
\end{itemize}

\section{Возможные сложности и решения}
\subsection{Чтение строк с пробелами}
При использовании \texttt{scanf} невозможно корректно принять строку с пробелами; решено читать строки через \texttt{fgets} и предварительно аккуратно обработать остаток ввода (удалить лишний перевод строки).

\subsection{Синхронизация состояния (загруженный файл)}
Требуется, чтобы пункт 2 работал только если ранее успешно выполнен пункт 1. Для этого введена переменная \texttt{current\_path}, которая обновляется только после успешного чтения файла в пункте 1. При ошибке загрузки путь очищается.

\subsection{Проверка пустого файла}
Пустой файл считается ошибкой согласно условию: размер файла вычисляется через \texttt{ftell}, и при размере меньше или равном нулю выводится \texttt{n/a}.

\subsection{Стилевая и мемориальная чистота}
Все динамические выделения освобождаются, обработчики ошибок возвращают корректное состояние. Для проверок стиля можно использовать \texttt{clang-format} с конфигурацией из \texttt{materials/linters}.

\section{Заключение и возможные улучшения}
Задача реализована в соответствии с требованиями: поддерживается чтение и запись строк с пробелами, корректная обработка ошибок и специальной команды \texttt{-1}. Для дальнейшего улучшения можно рассмотреть:
\begin{itemize}
  \item Поддержку путей с пробелами (сейчас путь читается как токен без пробелов — соответствует тестам задания).
  \item Добавить модуль тестирования с набором unit-тестов и скриптом автоматической проверки.
  \item Обработку больших файлов по частям (streaming) вместо полной загрузки в память.
  \item Более детальное логирование действий (module logger) — согласно следующим заданиям.
\end{itemize}

\section*{Как собрать и запустить}
В корне репозитория:
\begin{verbatim}
make cipher
./build/cipher
\end{verbatim}

\end{document}

\end{document}